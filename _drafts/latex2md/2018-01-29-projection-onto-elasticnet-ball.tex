\documentclass[]{article}




\usepackage[utf8]{inputenc}
\usepackage{hyperref}
\usepackage{makecell}
\usepackage{url}
\usepackage{caption}
\usepackage{subcaption}
\usepackage{array}
\usepackage{times,color}
\usepackage[rflt]{floatflt}
\usepackage{epsfig,times,graphicx}
\usepackage{amsmath,amssymb,amsopn,algorithm,algorithmic,theorem,float,bbm,bm,enumerate,color,multirow}
\usepackage{rotating}
\usepackage{array}
%\usepackage{slashbox}
\usepackage{makecell}
\usepackage{multirow}
\usepackage{hhline}
\usepackage{xspace}
\usepackage{mathtools}
\usepackage{bm}





% Matrices
\newcommand{\X}{\mathbf{X}}


% Vectors
%% English
\newcommand{\y}{\mathbf{y}}
\newcommand{\x}{\mathbf{x}}
\renewcommand{\v}{\mathbf{v}}
\renewcommand{\u}{\mathbf{u}}
\newcommand{\w}{\mathbf{w}}

%% Greek 
\newcommand{\bbeta}{\boldsymbol{\beta}}


\newcommand{\norm}[2]{\|#1\|_{#2}}
\newcommand{\be}{\[ \begin{alignedat}{2}}
\newcommand{\ee}{\end{alignedat} \]}
\newcommand{\beq}{\begin{equation}}
\newcommand{\eeq}{\end{equation}}
\newcommand{\argmin}{\mathop{\mathrm{argmin}}\limits}   
\newcommand{\argmax}{\mathop{\mathrm{argmax}}\limits}   
\newcommand{\prox}{\text{\bf prox}}
\newcommand{\sgn}{\text{\bf sgn}}
%\newcommand{\nr}{\nonumber} %Don't use it on the web!!!


\newcommand{\indic}{\mathbbm{1}}
\newcommand{\one}{{\bf1}}




% Sets
\newcommand{\cL}{{\cal L}}
\newcommand{\cS}{{\cal S}}


\title{Projection onto Elastic Net Ball}

\begin{document}
	
How can we project to an arbitrary norm ball? 

\section{Major Lessons}
I think the general way of doing this is to go for proximal operators. Proximal operator of a closed proper convex function $f(\cdot)$ is defined as: 
\beq 
\prox_f (\v) = \argmin_\x \frac{1}{2} \norm{\v - \x}{2}^2 + f(\x) 
\eeq 
Note that the objective of the above optimization is strongly convex, so there is a unique minimizer for it. 

\subsection{Generalizing Projection}
Proximal operator is a generalization of projection in the following sense. Remember the indicator of a set $\cS$, i.e.,
\beq 
\indic_\cS(\x) = 
\begin{cases}
	0& \x \in \cS \\
	\infty& \x \notin \cS \\ 
\end{cases}
\eeq 
Then the proximal operator reduces to Euclidian projection:
\beq 
\prox_f (\v) = \argmin_{\x\in \cS} \frac{1}{2} \norm{\v - \x}{2}^2 = \Pi_\cS(\v) 
\eeq 
\subsection{Generalizing Orthogonal Decomposition}
Let $\cS$ and $\cS^\perp$ be two orthogonal subspaces. Then for any vector $\v$ we have:
\beq 
\v = \Pi_{\cS}(\v) + \Pi_{\cS^\perp}(\v)  
\eeq 
We have the same decomposition for closed proper convex $f$ and its convex conjugate $f^*(\y) = \argmin_\x \x^T\y - f(\x)$:
\beq 
\v = \prox_f (\v) + \prox_{f^*} (\v)
\eeq 
The above result is known as {\bf Moreau Decomposition}. 

{\bf Proof:}
For the $\u = \prox_f (\v)$, from the optimality condition we should have:
\beq 
\v - \u \in \partial f(\u) \iff \u \in \partial f^*(\v - \u ) \iff (\v - \u) = \prox_{f^*}(\u)
\eeq 

Above, we used a useful property which relates subgradient of $f$ and $f^*$:
\beq 
\u \in \partial f(\v) \iff \v \in \partial f^*(\u) \iff f^*(\u) + f(\v) = \u^T \v 
\eeq 

\subsection{Moreau Decomposotion for a Norm}
Here we first need to following for an arbitrary norm $f(\x) = \norm{\x}{}$:
\beq 
f^*(\y) = \indic_{\norm{\cdot}{*} \leq 1}(\y)
\eeq 
where $\norm{\cdot}{*}$ is the daul norm. In other words, the convex conjugate of a norm is the indicator function of its dual norm. 

From the above equality and the Moreau decomposition we get: 
\be 
\v 
&= \prox_{\norm{\cdot}{}} + \prox_{\indic_{\norm{\cdot}{*} \leq 1}(\cdot)} 
\\ 
&= \prox_{\norm{\cdot}{}} + \Pi_{\indic_{\norm{\cdot}{*} \leq 1}} (\v)
\ee 
Above equality gives us a way to switch between proximal operator $\prox$ and projection $\Pi$.




\end{document}