\documentclass[]{article}




\usepackage[utf8]{inputenc}
\usepackage{hyperref}
\usepackage{makecell}
\usepackage{url}
\usepackage{caption}
\usepackage{subcaption}
\usepackage{array}
\usepackage{times,color}
\usepackage[rflt]{floatflt}
\usepackage{epsfig,times,graphicx}
\usepackage{amsmath,amssymb,amsopn,algorithm,algorithmic,theorem,float,bbm,bm,enumerate,color,multirow}
\usepackage{rotating}
\usepackage{array}
%\usepackage{slashbox}
\usepackage{makecell}
\usepackage{multirow}
\usepackage{hhline}
\usepackage{xspace}
\usepackage{mathtools}
\usepackage{bm}





% Matrices
\newcommand{\X}{\mathbf{X}}


% Vectors
%% English
\newcommand{\y}{\mathbf{y}}
\newcommand{\x}{\mathbf{x}}
\renewcommand{\v}{\mathbf{v}}
\renewcommand{\u}{\mathbf{u}}
\newcommand{\w}{\mathbf{w}}

%% Greek 
\newcommand{\bbeta}{\boldsymbol{\beta}}


\newcommand{\norm}[2]{\|#1\|_{#2}}
\newcommand{\be}{\[ \begin{alignedat}{2}}
\newcommand{\ee}{\end{alignedat} \]}
\newcommand{\beq}{\begin{equation}}
\newcommand{\eeq}{\end{equation}}
\newcommand{\argmin}{\mathop{\mathrm{argmin}}\limits}   
\newcommand{\argmax}{\mathop{\mathrm{argmax}}\limits}   
\newcommand{\prox}{\text{\bf prox}}
\newcommand{\sgn}{\text{\bf sgn}}
%\newcommand{\nr}{\nonumber} %Don't use it on the web!!!


\newcommand{\indic}{\mathbbm{1}}
\newcommand{\one}{{\bf1}}




% Sets
\newcommand{\cL}{{\cal L}}
\newcommand{\cS}{{\cal S}}


\title{Projection onto Elastic Net Ball}

\begin{document}
	
How can we project to the Elastic Net ball? 

\section{Major Lessons}
In the last post we saw that Moreau decomposition may give us a way on projecting onto norm balls. But the problem is that the elastic net is not a norm (why?). So to project on it we need another method. 

The way that John Duchi projects onto $l_1$-norm ball may be a good line of attack. Here are the steps that he uses to give a good algorithm for this:

\subsection{Projecting onto $l_1$-norm Ball}
\begin{itemize}
	\item 
\end{itemize}


\subsubsection{Projecting onto a Simplex}

\subsubsection{Why it Works}

\subsection{Projecting onto Elastic Net Ball}
Here is the projection problem:

\be
\w = \argmin_{\y} \frac{1}{2} \norm{\v - \y}{2}^2, \quad \norm{\y}{1} + \frac{1}{2} \norm{\y}{2}^2 \leq b.
\ee 
Like before, from the symmetry of the elastic net ball, we can consider the following problem and compute the solution of the original problem from $w_i = \sgn(v_i) u_i$:
\be 
\u = \argmin_{\x} \frac{1}{2} \norm{|\v| - \x}{2}^2, \quad \norm{\x}{1} + \frac{1}{2} \norm{\x}{2}^2 \leq b, \x \succeq 0 
\ee 
where $(|\v|)_i = |v_i|$.
Writing the Lagrangian: 
\be 
\cL = \frac{1}{2} \norm{|\v| - \x}{2}^2 + \alpha (\x^T \one + \frac{1}{2} \norm{\x}{2}^2 - b) + \bbeta^T \x  
\ee 
Taking derivative w.r.t. $x_i$ and setting it to zero:
\be 
u_i  = \frac{|v_i| - \alpha - \beta_i}{1 + \alpha}
\ee 
If $u_i >  0$ complementary slackness tells us that $\beta_i = 0$, so:
\be 
u_i  = \frac{|v_i| - \alpha}{1 + \alpha}
\ee 
Assume that we know the number of non-zero elements of the optimal solution $\u$ to be $\rho$. Therefore $\forall i \in [\rho]: |v_{(i)}| > \alpha$
Then we can write the feasibility of $\u$ as:
\be 
b 
&= \sum_{i=1}^{\rho} u_i + \frac{1}{2} \sum_{i=1}^{\rho} u_i^2 
\\ 
&= \frac{1}{1 + \alpha} \sum_{i=1}^{\rho} (|v_i| - \alpha) + \frac{1}{2(1 + \alpha)} (|v_i| - \alpha)^2
\\
&= \frac{1}{2(1 + \alpha)^2} \sum_{i=1}^{\rho} (2|v_i| + 2\alpha |v_i| - 2 \alpha - 2 \alpha^2) + (|v_i|^2 + \alpha^2 - 2 \alpha |v_i|)
\\
&= \frac{1}{2(1 + \alpha)^2} \sum_{i=1}^{\rho} \left[|v_i|^2  + 2|v_i| + 1 -  (\alpha^2 + 2 \alpha + 1) \right]
\\
&= \frac{1}{2(1 + \alpha)^2} \sum_{i=1}^{\rho} (|v_i| + 1)^2 -  \frac{\rho}{2}
\ee 
Assuming $\rho$ is known, we get the following for the $\alpha$:
\be 
\alpha = \sqrt{\frac{1}{2b + \rho} \sum_{i=1}^{\rho} (|v_i| + 1)^2} - 1
\ee 
Now, we claim that $\rho$ is the following:
\be 
\rho = \max \left\{j \in [p]: v_{(j)} - \frac{1}{j} \left(\sum_{i=1}^{j} \left[v_{(i)} + \frac{v^2_{(i)}}{2}\right] - b \right) > 0 \right\}
\ee 

Let's assume that we are wrong. Then the correct number of non-zero elements of the actual solution $\u^*$ is $\rho^*$ with the corresponding Lagrange multiplier $\alpha^*$. We know that still we have the following:
\be 
\forall i \leq \rho^*: u^*_{(i)} = \frac{|v_{(i)}| - \alpha^*}{1 + \alpha^*} > 0, \quad \alpha^* = \sqrt{\frac{1}{2b + \rho^*} \sum_{i=1}^{\rho^*} (|v_{(i)}| + 1)^2} - 1
\ee 
First consider the case of $\rho < \rho^*$. Note that in such a case, $\alpha^* < \alpha$. 







































\end{document}